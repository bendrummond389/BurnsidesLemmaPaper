\documentclass[../main.tex]{subfiles}
\setlength{\parskip}{1em}

\begin{document}

    Upon researching this topic I have found that the notation used to discuss topics in this field vary, so in this section I will provide some definitions for the conventions I will be using.
    
    \textbf{Group Action}
    
    If $G$ is a group acting on a set $X$, then a group action is a group homomorphism $\phi$ from $G$ to the group of possible permutations of $X$ denoted $S(X)$.
    \begin{equation*}
    \phi: G \rightarrow S(X)
    \end{equation*}
    Rather than denoting the group action of some $g$ on an element $x$ as $\phi(g)(x)$ we will instead write $gx$.

    \textbf{Orbit}

    \begin{equation*}
    Orb(x) := \{gx:g \in G\}
    \end{equation*}

    \textbf{Fixed point}
    
    if an element if $G$ is a group acting on a set $X$ and we have that $gx = x$ then we say that $x$ is a fixed point of $g$ or that $g$ fixes $x$.

    \textbf{Stabilizer}
    
    The stabilizer of an element $x \in X$ is the subgroup of $G$ that fixes $x$. The notation and formula are are as follows:
    \begin{equation*}
    G_x := \{g \in G: gx = x\}
    \end{equation*}
    
    \textbf{Fix of A Group Element}
    The fix of a group element $g \in G$ is the set of elements in $X$ that are a fixed by $g$. In someways this can be thought of as the complement of a stabilizer and the following notation is reflective of this.
    \begin{equation*}
    X^g := \{x \in X: gx = x\}
    \end{equation*}
    
    
    
\end{document}