\documentclass[../main.tex]{subfiles}
\setlength{\parskip}{1.5em}

\begin{document}
    In this section we will be looking at a proof of Burnside's lemma; the proof presented closely follows the proof found on the Wolfram page on Burnside's lemma \cite{weisstein}.

    Let $G$ be a finite group that acts on the set $X$. Earlier we mentioned that $X^g$ and $G_x$ almost complementary, and there is a good reason for this. We know that
    :
    \begin{align*}
        \sum_{g \in G} \lvert {X^g} \rvert &= \sum_{g \in G} \lvert \{x \in X : gx = x\} \rvert\\
        \sum_{x \in X} \lvert {G_x} \rvert &= \sum_{x \in X} \lvert \{g \in G : gx = x\} \rvert
    \end{align*}
    
    Its not hard to see that these are the same sums over different indices which means
    
    \begin{equation*}
        \frac{1}{\lvert G \rvert} \sum_{g \in G} \lvert {X^g} \rvert = \frac{1}{\lvert G \rvert} \sum_{x \in X} \lvert {G_x} \rvert
    \end{equation*}
    
    Using the orbit stabilizer theorem we get
    
    \begin{equation*}
        \begin{split}
            \frac{1}{\lvert G \rvert} \sum_{g \in G} \lvert {X^g} \rvert & = \frac{1}{\lvert G \rvert} \sum_{x \in X}  \frac{\lvert G \rvert}{\lvert Orb(x) \rvert} \\
        & = \sum_{x \in X}  \frac{1}{\lvert Orb(x) \rvert}
        \end{split}
    \end{equation*}
    
    Since no $x \in X$ can be in more than one orbit, and the orbit of $x$ can't be empty we find that $X$ is the disjoint union of all its orbits denoted $X/G$. With this in mind we can break the sum into the following:
    
    \begin{equation*}
        \begin{split}
            \sum_{x \in X}  \frac{1}{\lvert Orb(x) \rvert} & = \sum_{Orb(x) \in X/G} \left( \sum_{x \in Orb(x)} \frac{1}{\lvert Orb(x) \rvert} \right) \\
        & = \sum_{Orb(x) \in X/G} {1} \\
        & = \lvert X/G \rvert
        \end{split}
    \end{equation*}
    
    We've arrived at Burnside's lemma.
    
    \begin{equation*}
        \frac{1}{\lvert G \rvert} \sum_{g \in G} \lvert {X^g} \rvert = \lvert X/G \rvert
    \end{equation*}

        


    
    
\end{document}
